% AER-Article.tex for AEA last revised 22 June 2011
\documentclass[AER]{AEA}

% The mathtime package uses a Times font instead of Computer Modern.
% Uncomment the line below if you wish to use the mathtime package:
%\usepackage[cmbold]{mathtime}
% Note that miktex, by default, configures the mathtime package to use commercial fonts
% which you may not have. If you would like to use mathtime but you are seeing error
% messages about missing fonts (mtex.pfb, mtsy.pfb, or rmtmi.pfb) then please see
% the technical support document at http://www.aeaweb.org/templates/technical_support.pdf
% for instructions on fixing this problem.

% Note: you may use either harvard or natbib (but not both) to provide a wider
% variety of citation commands than latex supports natively. See below.

% Uncomment the next line to use the natbib package with bibtex
\usepackage{natbib}

% Uncomment the next line to use the harvard package with bibtex
%\usepackage[abbr]{harvard}

% This command determines the leading (vertical space between lines) in draft mode
% with 1.5 corresponding to "double" spacing.
\draftSpacing{1.5}

% Pandoc citation processing


\usepackage{hyperref}

\begin{document}

\title{Mapping choice sets from micro to market frictions in macro}

% \author{Author1 and Author2\thanks{Surname1: affiliation1, address1, email1.
% Surname2: affiliation2, address2, email2. Acknowledgements}}


\author{
  Hans Martinez\thanks{
  Martinez: Western
University, \href{mailto:hmarti33@uwo.ca}{hmarti33@uwo.ca}.
  Hi!
}
}

\date{\today}
\pubMonth{03}
\pubYear{2021}
\pubVolume{1}
\pubIssue{1}
\JEL{A10, A11}
\Keywords{Macro, Micro, Cool new thing I'm doing}

\begin{abstract}
Recent research in applied choice theory has evidenced the relevance of
the consideration sets of decision-makers. Yet, until now, there's been
a disconnect between these novel applications and the macro
applications. We show how to map choice sets to market frictions by
making then minimal necessary assumptions and how to use these novel
developments in a macro setting. Using data from the US Bureau of Labor
Statistics, we apply these techniques to a job searching problem in
which it is costlier for applicants to look for jobs that are farther
away from their home/current job location/current field of expertise. In
other words, the cost of accessing a bigger consideration set increases
with the distance from the current state.
\end{abstract}


\maketitle

\hypertarget{introduction}{%
\section*{Introduction}\label{introduction}}
\addcontentsline{toc}{section}{Introduction}

This is my super cool paper. I'm awesome.

\end{document}
