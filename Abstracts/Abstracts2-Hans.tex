% AER-Article.tex for AEA last revised 22 June 2011
\documentclass[AER]{AEA}

% The mathtime package uses a Times font instead of Computer Modern.
% Uncomment the line below if you wish to use the mathtime package:
%\usepackage[cmbold]{mathtime}
% Note that miktex, by default, configures the mathtime package to use commercial fonts
% which you may not have. If you would like to use mathtime but you are seeing error
% messages about missing fonts (mtex.pfb, mtsy.pfb, or rmtmi.pfb) then please see
% the technical support document at http://www.aeaweb.org/templates/technical_support.pdf
% for instructions on fixing this problem.

% Note: you may use either harvard or natbib (but not both) to provide a wider
% variety of citation commands than latex supports natively. See below.

% Uncomment the next line to use the natbib package with bibtex
\usepackage{natbib}

% Uncomment the next line to use the harvard package with bibtex
%\usepackage[abbr]{harvard}

% This command determines the leading (vertical space between lines) in draft mode
% with 1.5 corresponding to "double" spacing.
\draftSpacing{1.5}

% Pandoc citation processing


\usepackage{hyperref}

\begin{document}

\title{From new micro to macro: Finding the sweet spot}

% \author{Author1 and Author2\thanks{Surname1: affiliation1, address1, email1.
% Surname2: affiliation2, address2, email2. Acknowledgements}}


\author{
  Hans Martinez\thanks{
  Martinez: Western
University, \href{mailto:hmarti33@uwo.ca}{hmarti33@uwo.ca}.
  Hi!
}
}

\date{\today}
\pubMonth{03}
\pubYear{2021}
\pubVolume{1}
\pubIssue{1}
\JEL{A10, A11}
\Keywords{Macro, Finance, Cool new thing I'm doing}

\begin{abstract}
Recent developments in applied micro theory (stochastic revealed
preferences) can test rationality non-parametrically and allow for full
heterogeneity in utility functions. In doing so, however, they have to
abandon the hope of recovering the distributions of agents and thus
can't be used in applied macro settings. We show how to recover the
agents' distribution by imposing minimal assumptions. We then apply this
method to a simple general equilibrium setting in which agents differ in
their risk aversion and have access to a set of assets with different
risks and yields. We then test the implications in the data and find
that the agent's heterogeneity accounts significantly for risk premia.
\end{abstract}


\maketitle

\hypertarget{introduction}{%
\section*{Introduction}\label{introduction}}
\addcontentsline{toc}{section}{Introduction}

This is my super cool paper. I'm awesome.

\end{document}
