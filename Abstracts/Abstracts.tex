% AER-Article.tex for AEA last revised 22 June 2011
\documentclass[AER]{AEA}

% The mathtime package uses a Times font instead of Computer Modern.
% Uncomment the line below if you wish to use the mathtime package:
%\usepackage[cmbold]{mathtime}
% Note that miktex, by default, configures the mathtime package to use commercial fonts
% which you may not have. If you would like to use mathtime but you are seeing error
% messages about missing fonts (mtex.pfb, mtsy.pfb, or rmtmi.pfb) then please see
% the technical support document at http://www.aeaweb.org/templates/technical_support.pdf
% for instructions on fixing this problem.

% Note: you may use either harvard or natbib (but not both) to provide a wider
% variety of citation commands than latex supports natively. See below.

% Uncomment the next line to use the natbib package with bibtex
\usepackage{natbib}

% Uncomment the next line to use the harvard package with bibtex
%\usepackage[abbr]{harvard}

% This command determines the leading (vertical space between lines) in draft mode
% with 1.5 corresponding to "double" spacing.
\draftSpacing{1.5}

% Pandoc citation processing


\usepackage{hyperref}

\begin{document}

\title{Testing market power non-parametrically for multi-product firms}

% \author{Author1 and Author2\thanks{Surname1: affiliation1, address1, email1.
% Surname2: affiliation2, address2, email2. Acknowledgements}}


\author{
  Hans Martinez\thanks{
  Martinez: Western
University, \href{mailto:hmarti33@uwo.ca}{hmarti33@uwo.ca}.
  Special thanks to my momy.
}
}

\date{\today}
\pubMonth{03}
\pubYear{2021}
\pubVolume{1}
\pubIssue{1}
\JEL{A10, A11}
\Keywords{Market power, Industrial organization}

\begin{abstract}
Using the first-order conditions approach, convex analysis, and
stochastic process we model and test non-parametrically the market power
in the white appliances industry. Our method allows us to test firms'
market power by imposing minimal assumptions in contrast with current
literature. First, we get demand estimates and elasticities following
AK(2020). By varying ownership structure in the first-order conditions
of the firm's optimization problem, we can test if prices are optimal
for single-brand firms, current market structure, or a single
multi-brand monopolist. We apply this novel methodology to the white
appliances industry in the US. We find that market power accounts very
little for pricing, but it is product differentiation that matters.
\end{abstract}


\maketitle

\hypertarget{introduction}{%
\section*{Introduction}\label{introduction}}
\addcontentsline{toc}{section}{Introduction}

This is my super cool paper. I'm awesome.

\end{document}
